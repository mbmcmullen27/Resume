%============================================================================%
%
%	DOCUMENT DEFINITION
%
%============================================================================%

%we use article class because we want to fully customize the page and don't use a cv template
\documentclass[10pt,A4]{article}	


%----------------------------------------------------------------------------------------
%	ENCODING
%----------------------------------------------------------------------------------------

% we use utf8 since we want to build from any machine
\usepackage[utf8]{inputenc}		

%----------------------------------------------------------------------------------------
%	LOGIC
%----------------------------------------------------------------------------------------

% provides \isempty test
\usepackage{xstring, xifthen}

%----------------------------------------------------------------------------------------
%	FONT BASICS
%----------------------------------------------------------------------------------------

% some tex-live fonts - choose your own

%\usepackage[defaultsans]{droidsans}
%\usepackage[default]{comfortaa}
%\usepackage{cmbright}
%\usepackage[default]{raleway}
%\usepackage{fetamont}
%\usepackage{quattrocento}
%\usepackage[default]{gillius}     % gillius or raleway IWONA
%\usepackage[light,math]{iwona}
%\usepackage[thin]{roboto} 

% set font default
\renewcommand*\familydefault{\sfdefault} 	
\usepackage[OT1]{fontenc}

% more font size definitions
\usepackage{moresize}

%----------------------------------------------------------------------------------------
%	FONT AWESOME ICONS
%---------------------------------------------------------------------------------------- 

% include the fontawesome icon set
\usepackage{fontawesome}

% use to vertically center content
% credits to: http://tex.stackexchange.com/questions/7219/how-to-vertically-center-two-images-next-to-each-other
\newcommand{\vcenteredinclude}[1]{\begingroup
\setbox0=\hbox{\includegraphics{#1}}%
\parbox{\wd0}{\box0}\endgroup}

% use to vertically center content
% credits to: http://tex.stackexchange.com/questions/7219/how-to-vertically-center-two-images-next-to-each-other
\newcommand*{\vcenteredhbox}[1]{\begingroup
\setbox0=\hbox{#1}\parbox{\wd0}{\box0}\endgroup}

% icon shortcut
\newcommand{\icon}[3] { 							
	\makebox(#2, #2){\textcolor{maincol}{\csname fa#1\endcsname}}
}	

% icon with text shortcut
\newcommand{\icontext}[4]{ 						
	\vcenteredhbox{\icon{#1}{#2}{#3}}  \hspace{2pt}  \parbox{0.9\mpwidth}{\textcolor{#4}{#3}}
}

% icon with website url
\newcommand{\iconhref}[5]{ 						
    \vcenteredhbox{\icon{#1}{#2}{#5}}  \hspace{2pt} \href{#4}{\textcolor{#5}{#3}}
}

% icon with email link
\newcommand{\iconemail}[5]{ 						
    \vcenteredhbox{\icon{#1}{#2}{#5}}  \hspace{2pt} \href{mailto:#4}{\textcolor{#5}{#3}}
}

%----------------------------------------------------------------------------------------
%	PAGE LAYOUT  DEFINITIONS
%----------------------------------------------------------------------------------------

% page outer frames (debug-only)
% \usepackage{showframe}		

% we use paracol to display breakable two columns
\usepackage{paracol}

% define page styles using geometry
\usepackage[a4paper]{geometry}

% remove all possible margins
\geometry{top=1cm, bottom=0.5cm, left=1cm, right=1cm}

\usepackage{fancyhdr}
\pagestyle{empty}

% space between header and content
% \setlength{\headheight}{0pt}

% indentation is zero
\setlength{\parindent}{0mm}

%----------------------------------------------------------------------------------------
%	TABLE /ARRAY DEFINITIONS
%---------------------------------------------------------------------------------------- 

% extended aligning of tabular cells
\usepackage{array}

% custom column right-align with fixed width
% use like p{size} but via x{size}
\newcolumntype{x}[1]{%
>{\raggedleft\hspace{0pt}}p{#1}}%


%----------------------------------------------------------------------------------------
%	GRAPHICS DEFINITIONS
%---------------------------------------------------------------------------------------- 

%for header image
\usepackage{graphicx}

% use this for floating figures
% \usepackage{wrapfig}
% \usepackage{float}
% \floatstyle{boxed} 
% \restylefloat{figure}

%for drawing graphics		
\usepackage{tikz}				
\usetikzlibrary{shapes, backgrounds,mindmap, trees}

%----------------------------------------------------------------------------------------
%	Color DEFINITIONS
%---------------------------------------------------------------------------------------- 
\usepackage{transparent}
\usepackage{color}

% primary color
\definecolor{maincol}{RGB} { 0, 87, 194 }   %{ 225, 0, 0 }

% accent color, secondary
% \definecolor{accentcol}{RGB}{ 250, 150, 10 }

% dark color
\definecolor{darkcol}{RGB}{ 70, 70, 70 }

% light color
\definecolor{lightcol}{RGB}{245,245,245}


% Package for links, must be the last package used
\usepackage[hidelinks]{hyperref}

% returns minipage width minus two times \fboxsep
% to keep padding included in width calculations
% can also be used for other boxes / environments
\newcommand{\mpwidth}{\linewidth-\fboxsep-\fboxsep}
	


%============================================================================%
%
%	CV COMMANDS
%
%============================================================================%

%----------------------------------------------------------------------------------------
%	 CV LIST
%----------------------------------------------------------------------------------------

% renders a standard latex list but abstracts away the environment definition (begin/end)
\newcommand{\cvlist}[1] {
	\begin{itemize}{#1}\end{itemize}
}

%----------------------------------------------------------------------------------------
%	 CV TEXT
%----------------------------------------------------------------------------------------

% base class to wrap any text based stuff here. Renders like a paragraph.
% Allows complex commands to be passed, too.
% param 1: *any
\newcommand{\cvtext}[1] {
	\begin{tabular*}{1\mpwidth}{p{0.98\mpwidth}}
		\parbox{1\mpwidth}{#1}
	\end{tabular*}
}

%----------------------------------------------------------------------------------------
%	CV SECTION
%----------------------------------------------------------------------------------------

% Renders a a CV section headline with a nice underline in main color.
% param 1: section title
\newcommand{\cvsection}[1] {
	%\vspace{1pt}  % down from 14 originially
	\cvtext{
		\textbf{\LARGE{\textcolor{darkcol}{\uppercase{#1}}}}\\[-3pt] % from -3
		\textcolor{maincol}{ \rule{0.1\textwidth}{2pt} } \\
	}
	\vspace{-8pt}
}
%----------------------------------------------------------------------------------------
%	META SKILL
%----------------------------------------------------------------------------------------

% Renders a progress-bar to indicate a certain skill in percent.
% param 1: name of the skill / tech / etc.
% param 2: level (for example in years)
% param 3: percent, values range from 0 to 1
\newcommand{\cvskill}[3] {
	\vspace{-5pt}
	\begin{tabular*}{1\mpwidth}{p{0.72\mpwidth}  r}
 		\textcolor{black}{\textbf{#1}} & \textcolor{maincol}{#2}\\
	\end{tabular*}%
	
	\hspace{4pt}
	\begin{tikzpicture}[scale=1,rounded corners=2pt,very thin]
		\fill [lightcol] (0,0) rectangle (1\mpwidth, 0.15);
		\fill [maincol] (0,0) rectangle (#3\mpwidth, 0.15);
  	\end{tikzpicture}%
}


%----------------------------------------------------------------------------------------
%	 CV EVENT
%----------------------------------------------------------------------------------------

% Renders a table and a paragraph (cvtext) wrapped in a parbox (to ensure minimum content
% is glued together when a pagebreak appears).
% Additional Information can be passed in text or list form (or other environments).
% the work you did
% param 1: time-frame i.e. Sep 14 - Jan 15 etc.
% param 2:	 event name (job position etc.)
% param 3: Customer, Employer, Industry
% param 4: Short description
% param 5: work done (optional)
% param 6: achievements (optional)
% param 7: technologies (optional)
\newcommand{\cvevent}[7] {
	
	% we wrap this part in a parbox, so title and description are not separated on a pagebreak
	% if you need more control on page breaks, remove the parbox
	\parbox{\mpwidth}{
		\begin{tabular*}{1\mpwidth}{p{0.72\mpwidth}  r}
	 		\textcolor{black}{\textbf{#2}} & \colorbox{maincol}{\makebox[0.3\mpwidth]{\textcolor{white}{#1}}} \\
			\textcolor{maincol}{\textbf{#3}} & \\
		\end{tabular*}\\[8pt]
	
		\ifthenelse{\isempty{#4}}{}{
			\cvtext{#4}\\
		}
	}

	\ifthenelse{\isempty{#5}}{}{
		\vspace{2pt} % from 9
		{#5}
	}

	\ifthenelse{\isempty{#6}}{}{
		\vspace{2pt}
		\cvtext{\textbf{Projects:}}\\
		\vspace{-10pt}
		{#6}
	}

	\ifthenelse{\isempty{#7}}{}{
		\vspace{2pt}
		\cvtext{\textbf{"Technologies" used:}}\\
		{#7}
	}
	\vspace{10pt} % from 14
}

%----------------------------------------------------------------------------------------
%	 CV META EVENT
%----------------------------------------------------------------------------------------

% Renders a CV event on the sidebar
% param 1: title
% param 2: subtitle (optional)
% param 3: customer, employer, etc,. (optional)
% param 4: info text (optional)
\newcommand{\cvmetaevent}[4] {
	\textcolor{maincol} {\cvtext{\textbf{\begin{flushleft}#1\end{flushleft}}}}
	\ifthenelse{\isempty{#2}}{}{
	\textcolor{darkcol} {\cvtext{\textbf{#2}} }
	}
	\ifthenelse{\isempty{#3}}{}{
		\cvtext{{ \textcolor{darkcol} {#3} }} \\
	}

	\cvtext{#4}\\[2pt] % from 14
	\vspace{-10pt}
}

%============================================================================%
%
%
%
%	DOCUMENT CONTENT
%
%
%
%============================================================================%
\begin{document}
\columnratio{0.31} % from .31
\setlength{\columnsep}{2.2em}
\setlength{\columnseprule}{4pt}
\colseprulecolor{lightcol}
\begin{paracol}{2}
\begin{leftcolumn}

%---------------------------------------------------------------------------------------
%	META SKILLS
%----------------------------------------------------------------------------------------

\cvsection{CONTACT} \\

\icontext{MapMarker}{12}{Milwaukee WI,\\12345 Somewhere}{black}\\[6pt]
\icontext{MobilePhone}{12}{920-277-8465}{black}\\[6pt]
\iconemail{Envelope}{12}{mbmcmullen27@gmail.com}{mbmcmullen27@gmail.com}{black}\\[6pt]
\iconhref{Github}{12}{@mbmcmullen27}{https://github.com/mbmcmullen27}{black}\\[6pt]

\cvsection{LANGUAGES} \\

\cvskill{JavaScript} {5+ yrs} {0.95} \\[-2pt]

\cvskill{Regex *} {5+ yrs} {0.95} \\[-2pt]

\cvskill{C} {5+ yrs} {0.77} \\[-2pt]

\cvskill{Java} {5+ yrs} {0.83} \\[-2pt]

\cvskill{Python} {5+ yrs} {0.68} \\[-2pt]

\cvskill{Go} {1+ yrs} {0.59} \\[-2pt]

\cvskill{Bash} {1+ yrs} {0.85} \\[-2pt]

\cvskill{Terraform} {\textless\hspace{1pt} 1yr} {0.68} \\[-2pt]
 
\cvskill{WebAssembly} {\textless\hspace{1pt} 1yr} {0.5} \\[-2pt]

\cvskill{MANY MORE} {\hspace{-28pt}VERY SOON} {1} \\[-2pt]

\textcolor{darkcol}{* I know you're probably thinking \\"Regex? That's not a programming language!" I included it anyway as a formal grammar because I like it, and I think its an important skill. \\}
%\vfill\null

%---------------------------------------------------------------------------------------
%	EDUCATION
%----------------------------------------------------------------------------------------
%\newpage
\cvsection{EDUCATION}
\cvmetaevent
{\vspace{-10pt}2014 - 2019}
{B. Computer Science}
{University of Wisconsin\\Milwaukee}
{}
\vspace{-10pt}
%---------------------------------------------------------------------------------------
%	HOBBIES
%----------------------------------------------------------------------------------------

\cvsection{Hobbies}
\cvmetaevent
{RaspberryPi}
{}
{}
{ - clustering them\\  - putting Kubernetes on them\\  - putting hats on them}

\cvmetaevent
{Reading}
{}
{}
{
\iconhref{Link}{5}{Art of WebAssembly}{https://github.com/mbmcmullen27/artOfWasm}{black}\\ 
\iconhref{Link}{5}{Crafting Interpreters}{https://github.com/mbmcmullen27/lox}{black} \\ 
\iconhref{Link}{5}{Guide to Ncurses}{https://github.com/mbmcmullen27/ncurses}{black} \\
}

\cvmetaevent
{Online Classes}
{}
{}
{ - Udemy CKA prep course}

\end{leftcolumn}
\begin{rightcolumn}
%---------------------------------------------------------------------------------------
%	TITLE  HEADER
%----------------------------------------------------------------------------------------
\fcolorbox{white}{darkcol}{\begin{minipage}[c][3cm][c]{1\mpwidth}
	\begin {center}
		\HUGE{ \textbf{ \textcolor{white}{ \uppercase{ Michael McMullen } } } } \\[-24pt]
		\textcolor{white}{ \rule{0.1\textwidth}{1.25pt} } \\[4pt]
		\large{ \textcolor{white} {DevOps Engineer++} }
	\end {center}
\end{minipage}} \\[14pt]
\vspace{-12pt}

%---------------------------------------------------------------------------------------
%	PROFILE
%----------------------------------------------------------------------------------------

%\cvsection{About}

\cvtext{\vspace{5pt}DevOps engineer with an unhealthy fascination with code, its structure, and its execution.\\

My personal experience spans a variety of languages and frameworks and I believe strongly in accessibility, interoperability, and compatibility.\\

Reading, writing, learning languages. \\
Syntax and Strangeness. Deep dives only.\\ 
I love all software equally and I desparately need to know how it works. \\
Looking for excuses to scan, tokenize and parse. \\
Computering by day and also by night too... \\}

%---------------------------------------------------------------------------------------
%	WORK EXPERIENCE
%----------------------------------------------------------------------------------------
\vspace{10pt}
\cvsection{EXPERIENCE}
\vspace{10pt}
\cvevent
	{Jun 2020 - PRESENT}
	{DevOps Engineer, Alegeus}
	{Research and Development\vspace{-15pt}}
	{\vspace{-15pt}Containerization... Optimization... Automation... Awesome.\vspace{-15pt}}
	{\cvlist{
		\item Middle tier api management, deployment, and architecture consult
		\item Kubernetes cluster creation, administration, and maintenance
		\item Azure Cloud resource creation and management
		\item Monitor and maintain existing infrastructure
	}}
	{\cvlist {
		\item Migrated mannually developed and deployed Azure Cloud resources and dotnet middle tier api, inherited from a vendor, to Terraform managed infrastructure as code while maintaining live production architecture.
		\item Converted the scrum team's manual deployment process to git flow with Azure DevOps pipelines
		\item Wrote JavaScript and Bash utilities to bridge the gap between Terraform and Azure cli during the Terraform migration
	}}
	{}


\cvevent
	{May 2019 - Jun 2020}
	{Sustaining Engineer, Alegeus}
	{Tier III Support}
	{On call support engineer for consumer directed healthcare administration application written in dotnet. Responsible for addressing defects and for making tickets go away.}
	{\cvlist{
		\item Research and document defects in legacy dotnet web application
		\item SQL server database cleanup post defect resolutions
		\item Monitor system health and availability
		\item (Extra Credit) Circulated SQL scripts to generate common regular expressions for file manipulation tasks regularly performed by support staff
		\item (Extra Credit) Developed and maintained a small vscode extension for common SQL tasks, in an effort to get engineers to stop using SSMS and notepad++\\ \url{https://github.com/mbmcmullen27/AdsSqlCopy}
	}}
	{}
	{}

\end{rightcolumn}
\end{paracol}
\end{document}